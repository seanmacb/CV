\documentclass{resume} % Use the custom resume.cls style
\usepackage{textcomp}
\usepackage[left=0.75in,top=0.6in,right=0.75in,bottom=0.6in]{geometry} % Document margins
\newcommand{\tab}[1]{\hspace{.2667\textwidth}\rlap{#1}}
\newcommand{\itab}[1]{\hspace{0em}\rlap{#1}}
\usepackage{ragged2e}
\justifying
\usepackage{hyperref}
\usepackage{multicol}
\usepackage{enumitem}
\setlist[description]{leftmargin=\parindent,labelindent=\parindent}
\name{Sean MacBride} % Your name
% \address{40 Wildwood Drive \\ Burlington, VT 05408} % Your address
%\address{123 Pleasant Lane \\ City, State 12345} % Your secondary addess (optional)
\address{\href{tel:18029224673}{(802) 922 4673} \\ \href{mailto:Sean.P.MacBride@gmail.com}{E-Mail} \\ \href{https://github.com/seanmacb}{GitHub} \\ \href{https://www.linkedin.com/in/sean-macbride/}{LinkedIn}} % Your phone number and email %% RIT STUFF
% Add website when it's finished
\usepackage{seqsplit}
\begin{document}

\begin{rSection}{Education}

{\bf Wheaton College, Norton Massachusetts} \hfill {Aug 2016 - May 2020} \\
 B.A. in Physics with honors \hfill {GPA: 3.65 (cum laude)}
\\ Minor in Astronomy \hfill {Physics GPA: 3.76}
% \\ Studied Spring 2019 at University College London \hfill %{Astronomy GPA: 3.93}
\begin{itemize}\itemsep -6pt
 \item[$\star$] Active mentor, advisor, and collaborator of the Wheaton College physics club from 2016-2020
 \item[$\star$] Studied Spring 2019 at University College London
 \item[$\star$] Elected captain of the Wheaton College men's rugby team in the 2019-2020 season. Elected match secretary from 2017-2019. Elected club safety officer from 2017-2019.
\end{itemize}
\begin{tabular}{ @{} >{\hspace{5mm}\bfseries}l @{\hspace{2ex}} l }
Awards & Wheaton Centennial Grant for tremendous academic promise\\
& Boggess Family Foundation scholarship for achievements in physics
\end{tabular}
\end{rSection}

\begin{rSection}{Skills}
\begin{tabular}{ @{} >{\bfseries}l @{\hspace{2ex}} l }
Communication & Active listening, collaborative discussion, constructive feedback, empathy\\
Management & Strong time management, organized and reliable, composure under stress\\
Presentation & Articulate public speaker, effective communication in technical writing \\
Tools and Design & Electrical measurement, hand, and power tools, 3-D printers, electronics\\
Laboratory Skills & Lasers, sample preparation, spectrometers, precise instrumentation\\
Astronomy Operations & Optical telescopes, CCD cameras, dome and instrument maintenance\\
Computer Systems & Expert with Mac, Windows, and Linux systems, adept with MS Office\\
% Data Analysis &  NumPy, mathematical modeling, data pipeline management\\
Computer Languages &  Python (Primary), MATLAB, C++, Shell, Mathematica, \LaTeX, \href{https://github.com/seanmacb}{Git}\\
Python Packages & Jupyter, MatPlotLib, Seaborn, NumPy, Pandas, SciPy, Emcee\\


% AstroImageJ, APT
\end{tabular}
\end{rSection}

\begin{rSection}{Research Experience}
\begin{rSubsection}{University College London \& Wheaton College}{May 2019 - Present}{Honors Thesis Student}{Professor Am\'{e}lie Saintonge (UCL) \& Professor Dipankar Maitra (WC)}

\item Coordinated joint thesis between project between UCL and Wheaton College. Authored honors thesis detailing scientific rationale, results, and central conclusions of study. Presented to Wheaton and UCL faculty and students for critique and discussion. Received highest marks from faculty of both schools
\item Designed and implemented python data pipeline for use with derived data from xCOLD GASS, JINGLE, and SDSS galaxy surveys. Using Markov-chain Monte Carlo methods, constrained relationship of cold-gas and dust components of star-forming galaxies
% \item Presented results to Wheaton and UCL faculty and students for critique, questioning, and discussion. Received highest marks from faculty of both schools.
\item Identified effects of inclination-dependent reddening on relationship. Preparing paper for submission to the Astrophysical Journal
\end{rSubsection}

\begin{rSubsection}{University College London}{Jan 2019 - Mar 2019}{Group Lead}{Professor Thanh Nguyen}
\item Led research group of 8 UCL students in determining detection limits of thermochromic sheets in lateral flow assays using photo-thermal heating of magnetic nanoparticles
\item Resolved conflicts between group members and determined biophysics project structure. Coordinated laboratory logistics and safety measures between student group and the Royal Institute of London
\item Prepared ferromagnetic nanoparticle solution and samples for observation. Designed setup of infrared laser and sample apparatus. Obtained \& analyzed temperature data from a thermal camera for determination of optimal membrane type
\item Collaborated and authored report detailing group processes, results, \& analysis. Received highest marks from UCL faculty and Royal Institute staff
\end{rSubsection}

\newpage \begin{rSubsection}{Project P.A.N.O.P.T.E.S.}{Sep 2017 - Dec 2018}{External Collaborator}{James Synge (Google Cambridge)}
\item Communicated between professors at Wheaton and outside collaborators to install, operate, and maintain autonomous exoplanet telescope
\item Aligned P.A.N.O.P.T.E.S. unit according to polar-alignment procedure for observation in Wheaton College astronomy dome
\item Performed maintenance on the astronomy dome to ensure continuous observation of P.A.N.O.P.T.E.S.
\item Modified existing dome control code for weather-automated and manual remote operation of telescope
% \item Presented P.A.N.O.P.T.E.S. research at Northeast Astronomy Forum 2018
\end{rSubsection}

\begin{rSubsection}{Rutgers University New Brunswick}{May 2018 - Aug 2018}{REU Research Assistant}{Professor Carlton Pryor}
\item Improved proficiencies in Pandas, NumPy, SciPy, and other data tools to support investigations of dwarf satellite galaxies
\item Designed, implemented, and modified data pipeline for use with Gaia DR-2 data selections of dwarf satellite galaxies
\item Discovered small scale tidal tail around central core of dwarf satellite galaxy Carina
\item Prepared and presented research presentations to PI (Pryor) weekly. Prepared and delivered final presentation to summer research symposium and to professors and graduate students of the physics and astronomy department
% \item Presented Results at 234th meeting of the American Astronomical Society in St.  Louis, Missouri
\end{rSubsection}

\begin{rSubsection}{Wheaton College}{Aug 2017-Dec 2018}{Student astronomy projects}{Professor Dipankar Maitra}
\item 3D printed a mount for projection of sunlight on a 4.5" reflector telescope. Obtained and analyzed solar spectrum using a spectroscope. Observed spectral limb-reddening resulting from gas densities similar to the chromosphere. Authored report detailing methods, results, and analysis
\item Retrieved archived photometry from V404 Cygni outburst event obtained at Wheaton College in June 2015. Analyzed data using Aperture Photometry Tool. Identified two outburst events on the night of June 27, separated by roughly one hour. Authored report detailing methods, results, and analysis
\item Determined specifications of diffraction grating for use with DSLR camera. Obtained spectrum of Vega and Capella using grating and 30s exposure. Identified iron absorption features in both stars. Authored report detailing methods, results, and analysis
\item Retrieved x-ray binary data taken by MAXI instrument. Implemented Lomb-Scargle periodogram techniques to derive orbital periods of four different x-ray binary systems. Derived slight differences in period of binary 2S 1417-624. Authored report detailing methods, results, and analysis
\end{rSubsection}

\begin{rSubsection}{NASA Langley Research Center}{May 2017 - Aug 2017}{Summer Intern}{Dr. Brian Walsh}
\item Performed spectroscopy on Er:LuAG crystals using mid-infrared \& optical PerkinElmer spectrometers
\item Designed and operated a 2.1 \& 1.06 $\mu$m pump source for use in a future atmospheric-monitoring mid-infrared laser
\item Programmed a model to determine resonator stability based on resonator dimensions and elements
\item Determined resonator quality through spectroscopic tests. Optimized resonator setup from outcomes of spectroscopic tests and model predictions
\end{rSubsection}

\begin{rSubsection}{Wheaton College}{Dec 2016 - May 2017}{Undergraduate Assistant}{Professor John Collins}
\item Constructed setups for multiple Nd:YAG lasers, including soldering a new trigger circuit for a q-switch
\item Determined threshold energies for spontaneous emission using different resonator specifications
\item Exercised safe and appropriate use of voltmeters, oscilloscopes, and other electric and optical equipment
\end{rSubsection}

\end{rSection}

\newpage
\begin{rSection}{Outreach}

\begin{rSubsection}{Teaching Assistant}{August 2018 - December 2019}{Wheaton College Physics Department}{Norton, MA, USA}
\item Increased engagement of Introductory Physics I \& II students with in-class problems \& labs
\item Performed laboratory setup and breakdown for class of 40 students in accordance with schedule
\item Communicated student comprehension of specific topics to professors in an effort to increase classroom participation

\end{rSubsection}

\begin{rSubsection}{Physics Tutor}{August 2018 - December 2018}{Wheaton College Physics Department}{Norton, MA, USA}
\item Assisted intermediate physics students with problem sets, conceptual questions, and exam preparation
\item Increased engagement with struggling students by meeting outside of regular tutoring hours, communicated deficiencies to professors to optimize the classwork plan
\item Participated in tutor development meetings to enhance instruction and communication skills


\end{rSubsection}

\begin{rSubsection}{Observatory Guide}{August 2016 - May 2020}{Wheaton College Astronomy Department}{Norton, MA, USA}
\item Showcased features of different telescopes to local tour groups at Wheaton College Observatory
\item Operate and adjust telescopes to show appropriate stars, objects, and events based on weather and sky conditions
\item Educated young children and adults in locating objects, as well as providing knowledge about the objects being showcased

\end{rSubsection}

\end{rSection}

\begin{rSection}{Coursework}
\begin{multicols}{3}
    \begin{itemize}[label={}]\setlength\itemsep{-6pt}
        \item \textbf{Astronomy}
        \item[$\star$] The Universe
        \item[$\star$] Rocket Science
        \item[$\star$] Intro to Astrophysics
        \item[$\star$] Observational Astronomy
        \item[$\star$] Interstellar Physics (UCL)
        \item[$\star$] Physical Cosmology (UCL)

        \item \textbf{Physics}
        \item[$\star$] Classical Mechanics
        \item[$\star$]  Statistical Mechanics
        \item[$\star$] Group Project (UCL)
        \item[$\star$]  Electricity \& Magnetism
        \item[$\star$] Quantum Mechanics
        \item[$\star$] Fluid Mechanics

        \item \textbf{Math and CS}
        \item[$\star$] Robots, Games, \& Problem Solving
        \item[$\star$] Differential Equations
        \item[$\star$] Multivariable Calculus
        \item[$\star$] Scientific Computing
        \item[$\star$] Data Structures
        \item[$\star$] Linear Algebra

    \end{itemize}
    \end{multicols}

% \itab{\textbf{Astronomy Courses}} \tab{\textbf{Physics Courses}} \tab{\textbf{Other Courses}}
% \\ \itab{The Universe} \tab{Quantum Mechanics} \tab{Robots, Games, \& Problem Solving}
% \\ \itab{Intro to Astrophysics} \tab{Classical Mechanics} \tab{Scientific Computing}
% \\ \itab{Observational Astronomy} \tab{Statistical \& Thermal Physics} \tab{Data Structures}
% \\ \itab{Interstellar Physics} \tab{Electricity \& Magnetism} \tab{Differential Equations}
% \\ \itab{Physical Cosmology} \tab{Group Project (Biophysics)}  \tab{Multivariable Calculus}
% \\ \itab{Rocket Science} \tab{Fluid Mechanics} \tab{Linear Algebra}


\end{rSection} % probably only use this in

\begin{rSection}{Presentations}
\begin{rSubsection}{Poster Presentations}{}{}{}
\item Rutgers dwarf satellite galaxy research at Rutgers summer research symposium in August 2018
\item Rutgers dwarf satellite galaxy research at 234th meeting of the American Astronomical Society in June 2019
\item Project P.A.N.O.P.T.E.S. instrumentation at Northeast Astronomy Forum in April 2018
\end{rSubsection}
\begin{rSubsection}{Oral Presentations}{}{}{}
\item Honors thesis defense to peers and faculty of Wheaton College and University College London in May 2020
\item Rutgers dwarf satellite galaxy research to Wheaton College physics department in September 2018
\item Rutgers dwarf satellite galaxy research to Rutgers University physics and astronomy department in August 2018
\end{rSubsection}
\end{rSection}

\end{document}
